\chapter{项目成败分析}

总体上讲,本项目在与同期项目的比较中是较为成功的,但在我的眼中是有诸多失败之处的。

接下来便系统地分析成败。

\section{项目综合管理}

我这是第一次做项目经理,一开始完全是懵逼的。在我从前的印象中,项目经理就是一个一天到晚给程序员提要求然后被打的角色。现在发现并不是这样的。项目经理要负责项目大大小小各个方面的管理。跟程序员沟通的只是一个部分。

一上来就是\textbf{项目章程、项目管理计划}这样的作业,真的让人摸不着边,我只是怀着一个自觉得不错的想法就启动了项目。

我们的项目是一个自发组织的项目,没有上级领导的监管,也没有什么资金的来源,也没有工资与什么福利待遇。而且对项目的工期也没有一个很好的估计。总的来说,这都是因为我从前没有管理过项目,\textbf{缺乏经验}而导致的失败。

\section{项目范围管理}

这次项目的范围管理比较粗放,因为项目最初的设计、实现都由我个人完成,因此时常对大量变大放绿灯。

后来我发现这种鲁莽的变更给我带来了很多麻烦,比如说设计文档与实现的不一致等等。

实际上由于缺乏经验,一开始的设计总是不怎么靠谱的。

最后也没有完整实现我自己的设计。

\section{项目时间管理}

前期时间管理的相当粗放,一方面是因为有考试,我本人也无心管理;另一方面是计划经验不足,没有提高警惕。

本次项目甚至没有做甘特图来管理,到了后期我们才开始使用 GitHub Issue 来管理我们的活动。但亡羊补牢,为时不晚,在后期的时间管理中,我们取得了长足的进步。

\section{项目成本管理}

这次的成本,真是不好意思我都没有按照计划所说的工资表给大家发工资,当然我自己也没有工资拿。

因为大家都是学生,学生的劳动力真是不要钱呢。当然同时也有负面的影响,那就是学生的人力资源很难管理。

另外就是我们的工作场所,向软件学院申报了一个创新实验项目,然后免费地拿到了一个环境不错的实验室,节约了一大笔开支。

\section{项目质量管理}

出身于程序员,我对项目质量管理有着独特的见解:\textbf{代码风格要保持一致}、\textbf{要进行自动化的单元测试}、\textbf{代码文档要具有一致性}等。

我研究了很多相关的技术,最后敲定使用 ESLint, Mocha, JSDOC 等技术。

除了技术辅助之外,开发项目的同时还要进行重构来保证代码的质量。

在整个项目中,经历过两次大的重构,目前正在酝酿第三场重构。

\section{项目人力资源管理}

一开始的开发进度极慢,大家的积极性都不是很高。我有两个选择,一个是放弃我的组员,让他们自觉,自己独立完成,最后如果他们什么都没干就把他们批判一番;另一种是思考他们不行动的深层原因,并试图解决。

我想要开会,但发现除了食堂这样的公共场所,并没有能让我开会的地方。这可真愁人。好在天无绝人之路,刚好看到软件学院提供创新实验室,便申请了实验室的使用。

果然当人们有了私有的办公场所之后,积极性会有所提升。\textbf{因为这满足了他们的需求}。

然而这样的刺激并不能长久,很快人们就从“拥有了自己的办公室”中厌倦了。

这个时候我也束手无策,只能不断的鼓励他们,当然在这个过程中,我自己的不满度也在悄然积累。

\section{项目沟通管理}

这一次我在选组员的时候,特意选择关系比较近的人,比如室友。因为同在一个寝室,沟通成本会低得多。类似的,大多数组员都是我熟知的人,没有特意去选择专业中能力比较强的人。因为感觉可能沟通上会有问题。

一开始我们使用了QQ这种应该算是推送沟通的方式,后来发现频道容易串起来,沟通效率比较低。

在项目后期发现 GitHub Issue 提供了一种拉式沟通的途径,发现沟通效率提高了。人们的主动性也被调动起来了。

\section{项目风险管理}

这个项目风险主要有\textbf{项目工期中穿插大量考试}、\textbf{范围变更造成的进度回滚}、\textbf{部署环境出现偏差}等。

第一个风险对进度的影响是巨大的,对此我采取了将工期安排在考试刚结束,下一门考试还不紧张的间隙中,来降低风险;

第二个风险对进度的影响也很大,利用范围变更控制可以降低风险;

第三个风险我们采取采购外部服务的方式来进行风险转移。

\section{项目采购管理}

这个项目中的采购是值得一谈的,我们高效地采购了大量有效的资源来降低我们的成本。比如说向软院申请办公室,使用GitHub托管项目,使用Heroku来部署服务器,使用网易云服务来部署数据库等。我们几乎没有支付一分钱,就拿到了很多很多的可用资源。

\section{项目干系人管理}

这个项目的潜在干系人是可能使用产品的用户。因为我也是长期使用评测系统的用户,所以我就没有参考其他人的意见。

但是根据组织框架,如果我以后想要部署这个服务到学校,我必须征询学校相关领导的意见。

这是我项目干系人没有正确识别的一个问题。